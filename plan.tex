\documentclass[a4paper, 12pt]{article}
\usepackage[francais]{babel}
\usepackage[utf8]{inputenc}
\usepackage[T1]{fontenc}
\usepackage[margin=1.5cm]{geometry} % marge de la page
\usepackage{titlesec}
\usepackage{xcolor}
\titleformat*{\section}{\fontsize{14}{15}\bfseries\selectfont}
\titleformat*{\subsection}{\fontsize{13}{15}\bfseries\selectfont}
\titleformat*{\subsubsection}{\normalfont\normalsize\bfseries\selectfont}
\titleformat*{\paragraph}{\normalfont\normalsize\bfseries\selectfont}

\titlespacing\section{0pt}{12pt plus 4pt minus 2pt}{6pt plus 2pt minus 2pt}
\titlespacing\subsection{0pt}{12pt plus 4pt minus 2pt}{4pt plus 2pt minus 2pt}
\titlespacing\subsubsection{0pt}{12pt plus 4pt minus 2pt}{4pt plus 2pt minus 2pt}

\title{\vspace{-36pt}Code correcteur - Plan}
\author{G2S14 \small(\textcolor{orange}{Isabelle LIN}, \textcolor{cyan}{Filip SUDOL} et \textcolor{violet}{Hocine HAMMOUCHE})\vspace{-4cm}}
\date{}

\begin{document}
	\maketitle
	\color{violet}
	\section*{\vspace{-16pt}Introdution/problématique}
		\begin{itemize}
			\item Problème possible lors de la transmission de messages / communication
			\item Solution possible : la redondance → introduction au code correcteur
		\end{itemize}
	\color{orange}
	\section{Code correcteur}
		\subsection{Code de parité}
			\subsubsection{Parité simple}
			Présentation des propriétés des codes à parité simple et exemple
			\subsubsection{Parité double}
			Présentation des propriétés des codes à parité double (différences avec parité simple) et exemple
	\color{violet}		
	\section{Somme de contrôle}
		Définition, utilisation et exemple
	\color{cyan}	
	\section{Application : Code de Hamming}
		\subsection{Notions importantes et illustrations}
			\begin{itemize}
				\item Nombre de bit nécessaire pour encoder un ensemble de mots
				\item Poids de Hamming : $w_{H}(mot)$
				\item Distance de Hamming $d_{H}$
				\item Distance minimale $d_{min}$
					\begin{itemize}
						\item Si $d_{min} = k + 1$ alors peut détecter k erreurs.
						\item Si $d_{min} = 2 \times k + 1$ alors peut corriger k erreurs.
					\end{itemize}
			\end{itemize}	
		\subsection{Nombre minimum de bits de contrôle nécessaires pour détecter/corriger k erreurs lors de la transmission d’une information de m bits}
		Démonstration avec Borne de Hamming
		\color{orange}
		\subsection{Application}
			\subsubsection{Hamming (7, 4)}
			Description du code de Hamming (7, 4) et exemple de codage et décodage
			\subsubsection{Hamming (15, 11)}
			Description du code de Hamming (15, 11) et exemple de codage et décodage
	\color{violet}
	\section*{\vspace{6pt}Conclusion}
	
	\color{black}
	\section*{Annexe}
	\subsection*{Impémentation du code de Hamming}
		Une application graphique en Python qui permet à l'utilisateur de rentrer du texte et l'encode en binaire avec les bits de parité nécessaire et de décoder un texte.

		Un jeu qui consiste à retrouver le bit erroné dans une matrice correspondant à un code de Hamming. Le programme affiche un tableau de dimension 4 par 4, l'utilisateur sélectionne les bits dans l'ordre et les marque comme correcte ou incorrecte et à la fin il peut vérifier sa réponse.

	\section*{Références}
\end{document}